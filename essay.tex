\documentclass[a4paper,12pt]{article}
\usepackage[utf8]{inputenc}
\usepackage[spanish]{babel}
\usepackage{amsmath}
\usepackage{graphicx}
\usepackage{tocbibind}
\usepackage{graphicx}
\usepackage[absolute,overlay]{textpos} % para posicionar con coordenadas
\usepackage{tikz}
\usepackage{lmodern}
\usepackage[T1]{fontenc}
\usepackage{fancyhdr}
\usepackage{enumitem}
\usepackage{pdfpages}
\usepackage{xcolor} % Para definir colores
\usepackage[colorlinks=true, linkcolor=black, urlcolor=red, citecolor=red]{hyperref}


\pagestyle{fancy}
\fancyhf{} 
\renewcommand{\sectionmark}[1]{%
  \markright{#1}%
}
\renewcommand{\subsectionmark}[1]{}
\fancyhead[C]{\nouppercase{\rightmark}} 
\fancyfoot[C]{\thepage}
\renewcommand{\headrulewidth}{0.4pt}
\renewcommand{\footrulewidth}{0.4pt}




\begin{document}

\begin{titlepage}
    \centering
    \includepdf[pages=1, scale=1.1]{images/front.pdf}

    \vspace{1cm}
    \newpage
    
    {\Huge \textbf{Linus Torvalds y el desarrollo de Linux}}\\
    \vspace{0.5cm}
    
    {\Large José Antonio Concepción Alvarez \\  José Miguel Zayas Pérez \\ Grupo: 411 y 412}
    
    {\large \today}
   \thispagestyle{empty} 
\end{titlepage}

\renewcommand{\refname}{Bibliografía y Anexos}



\begin{centering}
    \vspace{2cm}
    \textbf{Resumen}\\
    Este ensayo explora la vida y las contribuciones de Linus Torvalds, el
    creador del núcleo Linux y Git, y su impacto en la historia de la
    computación. Se presenta un análisis de su relevancia en el desarrollo del
    software libre y de código abierto, destacando cómo su trabajo ha
    transformado el desarrollo colaborativo en la industria del software. A
    través de un examen de los antecedentes históricos, el nacimiento y
    evolución de Linux, y su influencia en la infraestructura de internet y
    dispositivos móviles, se argumenta que la influencia de Torvalds va más allá
    de la creación de software, abarcando un legado de colaboración y
    transparencia en el desarrollo tecnológico. Finalmente, se reflexiona sobre
    su impacto en el futuro de la tecnología y la filosofía del código abierto.
\end{centering}
\newpage

\tableofcontents



\newpage
\hypersetup{linkcolor=blue, urlcolor=blue, citecolor=blue}


\begin{quote}
\textbf{\textit{El software libre es un asunto de libertad, no de precio.}} 
Richard Altman
\end{quote}


\section{Introducción}

En la vasta cronología de la historia de la computación, pocos nombres han
dejado una huella tan profunda y duradera como el de Linus Torvalds. Reconocido
mundialmente como el creador del núcleo del sistema operativo Linux y del
sistema de control de versiones distribuido Git, Torvalds ha sido una figura
clave en la evolución del software moderno. Su obra no solo ha influido en la
infraestructura tecnológica global, sino que también ha catalizado una
transformación en la manera en que el software es desarrollado, compartido y
perfeccionado.

El núcleo Linux, iniciado como un proyecto personal en 1991, se ha convertido en
el corazón de millones de sistemas, desde servidores y supercomputadoras hasta
dispositivos móviles. A su vez, Git, concebido inicialmente para gestionar el
desarrollo del propio kernel de Linux, se ha posicionado como la herramienta
estándar para el control de versiones en proyectos de cualquier escala. Sin
embargo, más allá de estos hitos técnicos, la relevancia de Torvalds se extiende
a un plano más amplio: su trabajo ha redefinido las dinámicas del desarrollo
colaborativo, sentando las bases del movimiento de código abierto y fomentando
una cultura de cooperación, transparencia y \hyperlink{meritocracia}{meritocracia técnica}.

Este trabajo abordará la influencia de Linus Torvalds en el desarrollo de
software, el crecimiento de la comunidad de Open Source, la colaboración a gran
escala y la consolidación de un modelo de innovación abierto que ha transformado
profundamente la industria tecnológica y la cultura digital contemporánea.  
\newpage

% Sección 2
\section{Antecedentes} 

\subsection{Sistemas Operativos antes de la llegada de Linux}
A finales del siglo XX, el panorama de los sistemas operativos estaba marcado
por una diversidad de plataformas que reflejaban tanto los avances tecnológicos
como las limitaciones económicas y de acceso al software. Entre ellos,
UNIX, lanzado en 1969, emergió como un pilar fundamental en
entornos de alto rendimiento. Para inicios de los años noventa, se había
consolidado como el sistema operativo dominante en supercomputadoras y
servidores, siendo adoptado por gigantes tecnológicos como \hypertarget{ibm}{IBM},
\hyperlink{sun}{Sun} y \hyperlink{apple}{Apple} .  No obstante, su alto costo y la
fragmentación en múltiples variantes —derivadas de la base de código de
\hyperlink{att}{AT\&T} (System V) y de la Universidad de California-Berkeley
(\hyperlink{bsd}{BSD})— lo convertían en una opción inaccesible para la mayoría
de los estudiantes y usuarios particulares.

En paralelo, \hyperlink{msdos}{MS-DOS} se convirtió en el estándar de facto en el
mercado de computadoras personales. Al ser incluido en casi todas las máquinas
compatibles con IBM, permitió a \hyperlink{microsoft}{Microsoft}
alcanzar un dominio total en el mercado de sistemas operativos hacia 1991,
precisamente el año en que Linus Torvalds publicó la primera versión de prueba
del núcleo Linux. MS-DOS sirvió como el vínculo crucial entre
el hardware y el software, en una era anterior al auge de las interfaces
gráficas como Macintosh y Windows.

Antes de esta hegemonía de Microsoft, otro sistema operativo había tenido un
papel destacado: \hyperlink{cpm}{CP/M} (Control Program for Microcomputers),
desarrollado por \hyperlink{digitalresearch}{Digital Research}. Popular entre las
máquinas de 8 bits, CP/M se instaló en millones de dispositivos,
alcanzando un estimado de 200 millones de copias vendidas. Sin embargo, su
protagonismo fue efímero frente al crecimiento explosivo de
MS-DOS.

Aunque su relevancia había disminuido con la llegada de las computadoras
personales, los sistemas operativos de mainframe como \hyperlink{os360}{OS/360}
de IBM, así como \hyperlink{bostos}{BOS y TOS}, tuvieron una
importancia notable en décadas anteriores, especialmente en el ámbito
corporativo y gubernamental, sentando muchas de las bases conceptuales que
influenciarían generaciones posteriores de software.

Para estudiantes y académicos, una alternativa fue MINIX, un
sistema operativo de código fuente abierto diseñado con fines educativos por
Andrew S. Tanenbaum.  Fue precisamente utilizando MINIX en la
Universidad de Helsinki que Linus Torvalds identificó sus limitaciones y decidió
iniciar el desarrollo de su propio sistema operativo, con una arquitectura más
robusta y eficiente. Su proyecto fue anunciado inicialmente en un grupo de
discusión de MINIX, marcando el inicio de lo que se
convertiría en uno de los mayores fenómenos del software libre.


\subsection{El desarrollo de software hasta el momento}

Antes de la irrupción de Linux y el ascenso del movimiento de código abierto, el
desarrollo de software se encontraba fuertemente arraigado en un modelo cerrado
y comercial. El software propietario era la norma dominante: un sistema en el
que las empresas o individuos mantenían el control exclusivo del código fuente,
su distribución y su explotación económica. En esta lógica, el software era
considerado un producto sujeto a propiedad intelectual, cuya protección era
indispensable para asegurar su rentabilidad.

Una de las características esenciales del software propietario era precisamente
su naturaleza de código cerrado. Los usuarios podían adquirir licencias de uso,
pero solo accedían a binarios o código objeto, sin posibilidad de revisar,
modificar o adaptar el código fuente. Esta restricción limitaba tanto la
comprensión profunda del software como la capacidad colectiva para mejorarlo o
adaptarlo a nuevas necesidades.

El modelo se sustentaba en un esquema de comercialización y licenciamiento.
Empresas como Microsoft construyeron imperios bajo este
enfoque, los sistemas como MS-DOS dominaban el mercado de
computadoras personales. Incluso soluciones más potentes como
UNIX, operaban bajo costosas licencias que reservaban el acceso
y las modificaciones a una élite corporativa o académica con recursos
suficientes.

El desarrollo en este contexto era interno y centralizado. Los equipos de
programación pertenecían a las propias empresas, y tanto la depuración como la
mejora de los productos eran procesos controlados y frecuentemente lentos y
costosos. La dependencia de recursos internos limitaba la velocidad de
innovación, dificultando la respuesta ágil a los cambios del entorno
tecnológico.

Durante los años 60, la creciente complejidad de los proyectos llevó a lo que se
conoció como la ``crisis del software". Iniciativas como el sistema OS/360 de IBM
expusieron las dificultades de gestionar proyectos a gran escala, lo que impulsó
la búsqueda de metodologías más estructuradas para el desarrollo. Esta época
también vio un auge en el diseño de nuevos lenguajes de programación, destinados
a distintos nichos: \hyperlink{cobolfortran}{COBOL y FORTRAN} dominaron en la
industria y la ciencia, mientras que \hyperlink{basic}{BASIC}, desarrollado en
Dartmouth, encontró un lugar central en las primeras microcomputadoras.
\hyperlink{pascal}{Pascal}, introducido en 1971, se convirtió en un lenguaje
clave en la educación formal de programación.

Otro cambio técnico crucial de la época fue la introducción del `time-sharing´,
que permitió compartir el poder computacional de grandes máquinas entre
múltiples usuarios. Este paradigma impulsó el acceso a la computación en
entornos académicos y favoreció la difusión de lenguajes como BASIC, que
democratizaron el uso de la programación en los albores de la informática
personal.

En este entorno surgió UNIX, desarrollado en \hyperlink{belllabs}{Bell Labs} a partir de 1969. Su
portabilidad, eficiencia y adecuación para el`time-sharing´ lo convirtieron en el
sistema operativo preferido en universidades y centros de investigación, y más
tarde en servidores y supercomputadoras. Sin embargo, su costo y el carácter
cerrado de sus licencias lo mantuvieron alejado del usuario promedio.

En contraste con esta hegemonía comercial y cerrada, comenzaban a emerger
comunidades que compartían software libremente, como la del Laboratorio de
Inteligencia Artificial del MIT. El desencanto con las restricciones del
software propietario llevó a figuras como \hyperlink{stallman}{Richard Stallman} a iniciar un
movimiento en defensa de la libertad del usuario. En 1985 fundó la Free Software
Foundation y lanzó el proyecto \hyperlink{gnu}{GNU}, cuyo objetivo era construir un sistema
operativo completamente libre, compatible con UNIX. Stallman consideraba el
secretismo del código como un atentado contra el conocimiento colectivo.
Paralelamente, iniciativas como la \hyperlink{bsd}{Berkeley Software Distribution (BSD)} ofrecían
versiones abiertas de UNIX, sentando precedentes fundamentales para el
desarrollo posterior de Linux.

\subsection{Las universidades y cuestiones filosóficas} Las universidades
desempeñaron un papel crucial en el desarrollo del software, especialmente en
sus inicios, donde profesores y estudiantes universitarios fueron actores clave
en el movimiento del software libre. En la década de 1970, muchas universidades
lideraban proyectos de programación y vendían sus resultados, adoptando
políticas que promovían la publicación de software como libre bajo la licencia
\hyperlink{gpl}{GPL}. Por ejemplo, el laboratorio de programación de Harvard no
permitía la instalación de programas cuyo código fuente no estuviera disponible
públicamente. Además, los gobiernos financiaron proyectos de desarrollo de
software, como el compilador GNU Ada, desarrollado por la Universidad de Nueva
York con fondos de las Fuerzas Aéreas de EE. UU., con la condición de que el
código se donara a la Free Software Foundation. Sin embargo, las políticas
gubernamentales de control de exportaciones y leyes como la \hyperlink{dmca}{DMCA
(Digital Millennium Copyright Act)} podían restringir la distribución de
software libre a nivel internacional.

Los códigos fuente cerrados generaban frustración entre los usuarios, ya que les
impedían aprender del software y corregir errores, así como adaptar el software
a necesidades específicas. Un problema común de la época es que las
instituciones, no podían modificar un software comercial por falta de acceso al
código fuente, lo que resultaba en meses de trabajo innecesario. Además, la
existencia de software propietario obstaculizaba la evolución natural del
software, ya que obligaba a los desarrolladores a comenzar desde cero en lugar
de modificar programas existentes. Esto también dificultaba el aprendizaje de
nuevos programadores, quienes no podían estudiar el código fuente de programas
complejos. En ese contexto, sistemas operativos como el \hyperlink{its}{Sistema
Incompatible de Uso Compartido (ITS)} del MIT, escritos en lenguaje
\hyperlink{assembler}{Assembler}, eran dependientes de arquitecturas específicas
y se volvían obsoletos cuando estas dejaban de fabricarse. 

Desde una perspectiva filosófica, el surgimiento del movimiento del software
libre representó mucho más que una simple alternativa técnica: fue una postura
ética frente a la estructura dominante del desarrollo informático. Tras décadas
de hegemonía del software propietario, donde el conocimiento era celosamente
guardado y la colaboración entre usuarios vista como una amenaza, emergió una
crítica profunda al modelo que convertía el software en un bien cerrado y
restrictivo. Richard Stallman y los primeros defensores del software libre
denunciaban que ocultar el código fuente era no solo injusto, sino inmoral,
llegando a calificarlo como un pecado y un crimen contra la humanidad. En este
marco, el software libre no se trataba meramente de gratuidad, sino de libertad:
libertad para estudiar, modificar, compartir y construir colectivamente. Frente
a la narrativa empresarial que equiparaba compartir con piratería, el movimiento
proponía una ética de cooperación, de empoderamiento del usuario y de
construcción social del conocimiento. Rechazando el aislamiento al que condenaba
el software propietario, esta filosofía rescataba la dimensión humana de la
tecnología, defendiendo el derecho de las personas a controlar las herramientas
que median su vida digital. Así, el software libre planteaba una alternativa
civilizatoria: una tecnología al servicio de la libertad, no del control.

\subsection{El usuario final}
La relación entre el usuario final y el software, en el contexto del modelo
propietario, fue marcada por la exclusión, la dependencia y la pasividad. En
este modelo dominante antes de la irrupción del software libre, el usuario era
un receptor pasivo del producto, sin derecho ni posibilidad de intervenir en su
funcionamiento interno. El código cerrado —la práctica de distribuir software
únicamente en forma de binarios ejecutables, sin acceso al código fuente—
impidió que los usuarios comprendieran, adaptaran o mejoraran los programas que
usaban. Esta opacidad no solo restringía el aprendizaje técnico, sino que
reforzaba una barrera entre quien produce conocimiento tecnológico y quien
simplemente lo consume.

El usuario, por tanto, no era un colaborador ni un co-creador del software, sino
un consumidor subordinado a las decisiones del proveedor. Las licencias de uso
establecían los términos en los que se podía operar el software, negando
explícitamente la redistribución, modificación o copia. Esta situación
consolidaba una relación unidireccional, donde el proveedor centralizaba el
poder técnico y comercial, y el usuario quedaba relegado a un rol de obediencia
operativa. Incluso en casos en que el usuario —como una institución bancaria—
requería una modificación funcional específica, la ausencia de acceso al código
fuente lo condenaba a depender del proveedor o a incurrir en procesos costosos y
prolongados.

Además, la relación se caracterizaba por una fuerte dependencia técnica y
económica: si surgía un error o necesidad, el usuario debía esperar a que el
proveedor respondiera. Y si esa necesidad no era considerada rentable,
simplemente se ignoraba. Esta lógica instrumental convirtió al software en una
herramienta cerrada y rígida, en lugar de una plataforma adaptable o
personalizable.

Más profundamente, el modelo propietario fomentó una relación de alienación
entre el usuario y el software: se utilizaba, pero no se comprendía; se confiaba
en él, pero no se podía transformar. En este contexto, la figura del usuario
como agente activo del conocimiento era anulada. Cualquier intento de compartir
o adaptar software era estigmatizado como ``piratería", borrando las nociones de
solidaridad y cooperación que habían caracterizado los primeros entornos
académicos de programación.

Este escenario de restricciones, dependencia y exclusión fue el caldo de cultivo
ideal para que un joven Linus Torvalds —frustrado por las limitaciones de los
sistemas existentes como MINIX y motivado por una visión más abierta del
desarrollo tecnológico— iniciara en 1991 el proyecto que daría origen a Linux.
En un contexto dominado por el control propietario y la pasividad del usuario,
su propuesta representó un giro radical hacia la colaboración, la transparencia
y la libertad en el uso del software.

% Sección 3
\newpage
\section{El Nacimiento y Evolución de Linux} 

\subsection{El desarrollo inicial de Linux como proyecto personal}

El surgimiento de Linux está marcado no solo por la curiosidad técnica de Linus
Torvalds, sino también por sacrificios económicos y una tenacidad poco común. En
1991, mientras cursaba Ciencia de la Computación en la Universidad de Helsinki,
Torvalds invirtió 3500 dólares, una suma considerable para un estudiante
finlandés de la época, en adquirir un ordenador con procesador Intel 80386 a 33
MHz y 4 MB de RAM. Planteaba que era una máquina poderosa, pero MINIX, el
sistema operativo que usaba, no aprovechaba su potencial y, aunque cuyo código
era accesible, prohibía modificaciones fuera del ámbito educativo llevándolo a
escribir sus propios programas desde cero, prescindiendo incluso del sistema
operativo.

Sus primeros experimentos fueron programados en ensamblador, logrando arrancar
la máquina desde un disquete y ejecutar dos procesos, uno para la entrada desde
el teclado y otro para la salida del módem, en sus propias palabras esto era ``un
juego de prueba para entender cómo interactuar con el hardware". El siguiente
desafío surgió cuando intentó descargar archivos de la universidad, necesitaba
un controlador de disco compatible con el sistema de archivos de MINIX. Tras
semanas de trabajo logró un driver funcional dándose cuenta de que estaba 
construyendo algo más grande que un solo emulador, estaba creando las bases de
un sistema operativo.

El 25 de agosto de 1991, Torlvads anunció su proyecto en un grupo del sistema
operativo MINIX y un mes después, subía al servidor \hyperlink{ftp}{FTP} de la universidad la
versión 0.01 de Linux: 10 000 líneas de código que incluían un shell básico, el
compilador \hyperlink{gcc}{GCC} y herramientas para gestionar procesos y memoria, invitando de
esta manera a la comunidad de usuarios de MINIX a probar, criticar y aportar
sugerencias. Aunque limitado, este lanzamiento sentó las bases de un modelo de
desarrollo revolucionario, sobre su decisión de compartir el código de forma
abierta Torlavds dijo:

\begin{quote}
    \textbf{\textit{“Si hubiera hecho Linux de pago, nadie lo habría
    adoptado. Sin contribuciones externas, el proyecto habría muerto en meses”}} 
\end{quote}

Este enfoque colaborativo, combinado con la arquitectura modular inspirada en
UNIX, permitió que Linux evolucionara a velocidad exponencial. Para diciembre de
1991, una comunidad incipiente ya había aportado controladores para tarjetas de
vídeo y soporte inicial de redes, transformando lo que comenzó como un hobby en
el embrión de un sistema operativo global.

\subsection{La transición de Linux a un proyecto colaborativo}

El modelo de desarrollo de Torvalds, basado en
\hyperlink{meritocracia}{meritocracia técnica} y transparencia, desafió las
jerarquías corporativas. Linus evaluaba parches enviados por una comunidad
creciente, que incluía desde estudiantes hasta ingenieros de empresas como IBM.
En 1994, con el lanzamiento de Linux 1.0, la primera versión estable, el sistema
demostró ser viable comercialmente. Torvalds presentó esta versión en la
Universidad de Helsinki, destacando dos principios UNIX que adoptó:

\begin{itemize}[label=$\bullet$, itemsep=0.5em]
    \item \textbf{Todo es un archivo}: dispositivos, procesos y datos se
    gestionaban como archivos, simplificando la interoperabilidad.     
    \item \textbf{Cada herramienta hace una sola cosa y la hace bien}:
    esta modularidad permitía combinar componentes sin redundancia.
\end{itemize}

Ante el torrente de contribuidores, organizados en listas de correo y servidores

FTP, se estableció una estructura flexible:
\begin{itemize}[label=$\bullet$, itemsep=0.5em]
    \item \textbf{Linus Torvalds}: ``el General", revisaría y aceptaría los parches que
    cumplen los estándares de calidad y coherencia arquitectónica. 
    \item \textbf{Dave Miller}: responsable de la calidad, descartaría correcciones
    innecesarias y envíaría a Linus solo las más pulidas.  
    \item \textbf{Ted Ts’o}: principal promotor, se encargaría de las relaciones
    públicas y la difusión en conferencias y empresas.  
    \item \textbf{Alan Cox}: mano derecha de Linus, coordinaría grandes bloques de
    funcionalidades y gestionaría la estructura interna del núcleo.
\end{itemize} 

Esta gobernanza plana garantizó que cada desarrollador tuviera voz, pero siempre
bajo la garantía de que Torvalds y su equipo tomarían la decisión final, como
ratificó Alan Cox:

\begin{quote}
    \textbf{\textit{“No queríamos guerras de ego; si alguien proponía una mejora válida, se
    implementaba”}} 
\end{quote}
\newpage

\subsection{El papel de la Licencia Pública General de GNU en la expansión de Linux}

En enero de 1992, Linux adoptó la Licencia Pública General de GNU \hyperlink{gpl}{(GPLv2)}, un
paso clave que aseguraba el \hyperlink{copyleft}{copyleft}: todo derivado debía permanecer libre y
promover las mismas libertades de uso, estudio, modificación y redistribución. 
Esta decisión, alineada con el proyecto NU de Richard Stallman, generó un
entorno de confianza para empresas y particulares, al impedir que versiones
privativas del kernel excluyeran a la comunidad.

Bajo la protección de la GPL surgieron decenas de distribuciones orientadas a
distintos públicos y casos de uso: \hyperlink{slackware}{Slackware}, \hyperlink{debian}{Debian}, 
\hyperlink{redhatcanonical}{Red Hat}, \hyperlink{suse}{SUSE} y muchas más
ofrecían instaladores ``todo en uno”, con sistemas de paquetes, utilidades de
administración y repositorios. Cada distribución integraba Linux en un
ecosistema completo, facilitando su adopción en universidades, empresas y
hogares.

Además, la obligación de redistribuir las modificaciones bajo la misma licencia 
generó un efecto de red: cuanto más colaboraban los desarrolladores, más
atractivo resultaba el conjunto de herramientas y bibliotecas disponibles,
acelerando el ciclo de innovación y asegurando que los avances se reciclaran a
favor de toda la comunidad.


\subsection{La evolución de Linux y su adopción en servidores e infraestructura de Internet}

Con el lanzamiento de Linux 1.0 la compatibilidad entre diferentes entornos de
hardware mejoró, leer un disquete o mostrar en pantalla datos generados en otros
sistema dejó de ser un reto gracias a la modularidad de los controladores y la
estandarización de interfaces. Millones de personas lo descargaban, demostrando
cada segundo, y de forma tangible, un postulado antes no tan obvio para los 
usuarios y programadores: Open Source podía ser más rentable y atractivo que el
modelo de negocio empleado por otras compañías como Microsoft.

A finales de los 2000, Linux se impuso en el mercado de servidores gracias al
stack LAMP (Linux, \hyperlink{apache}{Apache}, \hyperlink{mysql}{MySQL},
\hyperlink{php}{PHP}/\hyperlink{python}{Python}/\hyperlink{perl}{Perl}) y al
coste cero en licencias.  Grandes compañías web y proveedores de hosting
pusieron sus centros de datos, y luego sus nubes, sobre Linux. Para 2008 más del
60\% de los servidores en producción funcionaban con núcleo Linux, y desde 2018
todos los equipos del TOP500 en supercomputación lo utilizan, aprovechando su
escalabilidad y la 
capacidad de tunear el kernel para arquitecturas masivamente paralelas.

Con el tiempo incluso grandes empresas como \hyperlink{oracle}{Oracle}, \hyperlink{netscape}{Netscape}, Corel e \hyperlink{intel}{Intel},
empezaron a prestar atención a las nuevas tendencias, y cada vez más a menudo
consideraban a Linux como una alternativa real a Windows. Por ejemplo, IBM
destinó alrededor de mil millones de dólares a la mejora adicional de Linux y,
desde entonces, ha seguido siendo uno de sus principales “embajadores”.

El éxito de Torvalds también sirvió de inspiración a otros, como la empresa
Android, fundada en 2003 por Andy Ruben, Richie Miner, Nick Sears y Chris White.
Tomando como base el núcleo de Linux, lo modificaron ligeramente y obtuvieron un
nuevo sistema operativo para dispositivos móviles, que tiene en cuenta la
geolocalización y la configuración personal de sus propietarios. Unos años 
después Google compró Android llevando el kernel no solo al bolsillo de más de
3000 millones de usuarios, sino que amplió su presencia a tablets, televisores,
relojes inteligentes, entre otros dispositivos. Paralelamente, proyectos como
BusyBox, U-Boot, OpenWrt y Tizen demostraron la versatilidad de Linux en
dispositivos con recursos limitados, desde routers domésticos hasta 
electrodomésticos conectados y maquinaria industrial.

Linux, por su parte, también siguió prosperando. En 2014, su distribución más
popular, Ubuntu, informó que 22 millones de usuarios la utilizaban. Otras
distribuciones populares son Arch, orientada a usuarios avanzados; Manjaro,
basada en Arch pero más amigable con el usuario y Debian, considerada la 
distribución madre al ser una de las primeras y base de infinidad de
distribuciones. 

Actualmente Linux es respaldado por la mayoría de las grandes empresas del
sector, las que colaboran con su difusión ya sea trabajando en el núcleo,
proporcionando soluciones de software o preinstalando el sistema operativo,
algunas de ellas son: Intel, Google, \hyperlink{ibm}{IBM}, \hyperlink{amd}{AMD}, \hyperlink{valve}{Valve}, \hyperlink{dell}{Dell}, \hyperlink{lenovo}{Lenovo}, \hyperlink{asus}{Asus},
Fujitsu, Oracle, entre muchas otras. De igual manera el apoyo de compañías de
software también está presente, ya que, entre otras aplicaciones Java, Google
Earth, Adobe Reader, Adobe Flash y Yahoo! Messenger están disponibles para
Linux.  

\subsection{El impacto económico de Linux}

Linux logró la combinación perfecta, grandes reducciones de gastos a las
empresas de software y cuantiosas ganancias a los desarrolladores, los cuales
podían vender servicios de configuración de servidores para empresas, así como
ayudar a mantener su funcionamiento. 

Plataformas como Amazon Web Services (\hyperlink{aws}{AWS}), lanzada en 2006, se construyeron
sobre Linux, permitiendo a startups acceder a infraestructura escalable a una
fracción del costo tradicional. Compañías como \hyperlink{redhat}{Red Hat} y \hyperlink{canonical}{Canonical}, creadora de
Ubuntu, demostraron que el software libre podía ser rentable. Red Hat facturó
miles de millones ofreciendo soporte técnico y certificaciones, no el código en
sí. Este modelo inspiró a otras empresas, como Docker, que monetizaron servicios
en lugar de licencias.

Linux también influyó en la innovación en hardware y el Internet de las Cosas
(\hyperlink{iot}{IoT}), permitiendo el auge de dispositivos embebidos de bajo costo. Raspberry Pi
(2012), una computadora de 35 dólares con Linux, se convirtió en herramienta
educativa y prototipado industrial. De igual manera empresas como Tesla
integraron Linux en sus sistemas de infoentretenimiento, reduciendo dependencia
de proveedores externos. 

\subsection{El impacto social de Linux}

Linux ha supuesto una auténtica democratización del acceso a la tecnología. Al
ser gratuito y funcionar en hardware modesto, ha permitido que comunidades con
recursos limitados monten laboratorios de informática en escuelas rurales o
centros comunitarios, usando desde equipos reciclados hasta plataformas de bajo
coste. Proyectos educativos offline basados en Linux, llevan contenidos
formativos a regiones sin conectividad, reduciendo la brecha digital y 
fomentando la autoformación.

Más allá del acceso al software, Linux ha cristalizado una cultura de
colaboración global.  Eventos como FOSDEM, organizados por y para
desarrolladores voluntarios, reúnen cada año a miles de asistentes en un espacio
de aprendizaje abierto y gratuito. A su vez, grupos locales de usuarios,
hackathons y conferencias regionales refuerzan el espíritu de solidaridad
técnica: todos contribuyen, todos se benefician.

En el ámbito académico, Linux y las herramientas asociadas se han incorporado
como pilares de la enseñanza en sistemas operativos, redes y desarrollo de
software colaborativo. Universidades de todo el mundo incluyen cursos y
certificaciones en sus planes de estudio, asegurando que las nuevas generaciones
dominen no solo el uso, sino también la filosofía del software libre.

El impacto social de Linux también se refleja en iniciativas de software libre
destinadas a resolver problemas reales: desde OpenStreetMap, que utiliza
servidores Linux para ofrecer mapas gratuitos y actualizados por la comunidad,
hasta plataformas de e-salud y gestión pública basadas en tecnologías abiertas.
Estas soluciones han empoderado a ONG, administraciones locales y colectivos 
ciudadanos para desarrollar y adaptar sus propias herramientas sin depender de
proveedores comerciales.

La adopción de las ``cuatro libertades" (usar, estudiar, modificar y compartir)
ha transformado al usuario pasivo en un co-creador activo de su entorno digital.
Este cambio de rol impulsa movimientos de datos abiertos y transparencia
gubernamental, promoviendo una ciudadanía digital capaz de auditar procesos,
compartir conocimientos y colaborar en la construcción colectiva de servicios y
políticas públicas.


\newpage
\section{Linus Torvalds y el Desarrollo Colaborativo}

\subsection{Una nueva forma}

Cómo hemos visto, Linux fue principalmente un proyecto personal de Linus Torvalds.
Comenzó a trabajar en el kernel como un hobby, sin la intención de competir con
sistemas comerciales como UNIX ni de desarrollar algo tan ambicioso como GNU. Su
objetivo inicial era simplemente crear un sistema operativo funcional que
pudiera utilizar en su propia computadora y, de paso, ayudar a otros
estudiantes.

Un aspecto fundamental de la organización inicial del proyecto fue su apertura
al público desde el primer momento. Torvalds anunció su trabajo en un grupo de
noticias del sistema operativo MINIX, invitando a otros usuarios a probar su
sistema y a ofrecer comentarios. Esta transparencia y disposición a recibir
retroalimentación contrastaban fuertemente con los modelos de desarrollo
tradicionales, que solían mantenerse en secreto durante sus fases tempranas.

Otro elemento clave fue la decisión de licenciar Linux bajo la General
Public License, ya que esta, permitía a cualquier persona usar, modificar
y distribuir el código fuente de Linux de forma libre. Fue una decisión
estratégica que atrajo a una comunidad creciente de desarrolladores, quienes
comenzaron a colaborar en el proyecto de manera voluntaria y entusiasta,
cimentando las bases del ecosistema de software libre que conocemos hoy.

La organización del desarrollo en sí misma era inicialmente muy informal y sin
una jerarquía estricta. No había una estructura de gestión tradicional con jefes
supervisando a los desarrolladores. Cada miembro de la comunidad tenía derecho a
voz y voto, y sus propuestas eran consideradas. Sin embargo, Linus se
mantuvo como la figura central, tomando las decisiones finales sobre qué se
incluía, eliminaba o cambiaba en el núcleo/kernel. Esta figura de `general'
surgió, según el propio Torvalds, para evitar conflictos y asegurar la
eficiencia del desarrollo.  Con el tiempo, surgieron roles más definidos de
manera orgánica dentro de la comunidad.

Podemos señalar además algunos contrastes con respecto a los modelos existentes:

\begin{itemize}
    \item \textbf{Desarrollo centralizado vs. desarrollo distribuido:} 
        Los modelos tradicionales, como los utilizados por Microsoft en el
        desarrollo de Windows, se basaban en pequeños equipos de programadores
        internos que trabajaban en un código cerrado. Linux, en cambio, se
        desarrolló de manera distribuida con la colaboración de cientos e
        incluso miles de personas de todo el mundo.
    \item \textbf{Motivación económica vs. motivación comunitaria:}  En los
    modelos tradicionales, los desarrolladores generalmente trabajaban por una
    recompensa económica directa. En el inicio de Linux, muchos contribuyeron
    voluntariamente, motivados por el respeto de sus colegas y el deseo de
    mejorar el software.
    \item \textbf{Desarrollo planificado vs. desarrollo evolutivo:} Los modelos
    tradicionales a menudo seguían un plan de desarrollo rígido, similar a un
    proyecto de construcción. El desarrollo de Linux fue mucho más orgánico y
    evolutivo, creciendo y mejorando gracias a las contribuciones de la comunidad.
\end{itemize}

Todos estos mecanismos permitieron programadores de todo el mundo colaboraran
eficazmente en el proyecto Linux.

\subsection{Linus, líder técnico y coordinador}
Aunque el proyecto se caracterizó por su desarrollo distribuido y la
colaboración de miles de programadores voluntarios de todo el mundo, Torvalds
siempre mantuvo la autoridad final sobre el kernel de Linux.
Un profundo conocimiento del sistema le permitió dirigir su desarrollo y tomar
decisiones técnicas cruciales. Incluso con el crecimiento masivo del proyecto,
Torvalds se mantuvo como la única persona autorizada para fusionar las
contribuciones al proyecto principal. Esto le permitía seleccionar las mejores
ideas, decidir qué incluir, eliminar o modificar en el núcleo/kernel. 
Aunque no existía una jerarquía estricta dentro de la comunidad, Torvalds se
convirtió en la figura central en quien la comunidad confiaba más. Su estilo de
liderazgo se describió como el de un `dictador benevolente', porque mantenía el
control final para asegurar la coherencia y dirección del proyecto, pero lo
hacía con el consenso y el respeto de la comunidad. Según el propio Torvalds, su
método para gestionar el proyecto con cientos de miles de desarrolladores era
esperar a que la gente se ofreciera voluntariamente para hacerse cargo de
diferentes subsistemas, en lugar de delegar proactivamente.

\subsection{Problemas con BitKeeper, el surgimiento de Git}

A medida que más programadores se unían al proyecto Linux y comenzaban a
contribuir con mejoras y correcciones, la tarea de revisar y gestionar todas
estas aportaciones se volvió cada vez más compleja. En un principio,
Torvalds administraba el flujo de contribuciones principalmente a través del
correo electrónico, lo cual funcionaba mientras el volumen de cambios era
manejable. Sin embargo, con el crecimiento exponencial de la comunidad, se hizo
evidente la necesidad de una herramienta más eficiente.

La escala y la naturaleza distribuida del desarrollo del kernel de Linux exigían
un sistema de control de versiones más robusto que los disponibles en ese
momento.

Los sistemas de control de versiones más utilizados en la época, eran
mayoritariamente centralizados. Herramientas como \hyperlink{concurrent versions
system }{Concurrent Versions System } y \hyperlink{svn}{Subversion} dominaban el
panorama, sustentadas en un modelo donde un único repositorio central albergaba
todo el historial de cambios.  Aunque estos sistemas funcionaban de forma
aceptable en entornos de equipos pequeños o proyectos simples, se mostraban
deficientes al escalar hacia proyectos complejos, distribuidos y de rápida
evolución como el kernel de Linux.

Los sistemas de control de versiones centralizados presentaban serias
limitaciones: dependían de un único servidor (punto crítico de fallo), ofrecían
bajo rendimiento en operaciones clave, complicaban la resolución de conflictos
entre desarrolladores y no permitían trabajar sin conexión. Estas deficiencias
resultaron inviables para proyectos complejos y distribuidos como el kernel de
Linux, que requerían una infraestructura más ágil, descentralizada y robusta.

Antes de 2002, Linus Torvalds no utilizaba ninguna herramienta formal de control
de versiones para el desarrollo del kernel de Linux. Aunque existían
opciones mencionadas anteriormente, decidió no adoptarlas debido a sus
limitaciones.  Sin embargo, a medida que el proyecto crecía y se volvía más
complejo, la comunidad de desarrolladores del kernel insistía en la necesidad de
una herramienta que facilitara la gestión del código.

Es por esto que, en 2002, surge la alianza con BitKeeper; una alternativa
descentralizada de código propietario, que permitía a cada desarrollador
tener una copia completa del historial del proyecto y trabajar de forma local,
mejorando significativamente la eficiencia y autonomía del equipo, aunque a
costa de depender de una herramienta cerrada cuya licencia gratuita resultaría
temporal y conflictiva.

Desde el punto de vista tecnológico, BitKeeper era una herramienta ideal para el
desarrollo del kernel de Linux: distribuida, rápida y eficiente. Sin embargo, no
era software libre, lo que generó fuertes tensiones dentro de la comunidad.
Aunque Linus Torvalds valoraba su funcionalidad y había decidido adoptarlo,
enfrentó una creciente oposición por parte del núcleo del equipo del kernel, que
rechazaba depender de una herramienta propietaria.

Sin embargo, la relación entre la comunidad del software libre y BitMover era
tensa desde el inicio, por la misma naturaleza cerrada de BitKeeper. Aunque
ofrecía licencias gratuitas para uso no comercial en proyectos de código
abierto, esta concesión era frágil. En 2005, tras una serie de desacuerdos,
BitMover decidió retirar el acceso libre a BitKeeper para los desarrolladores
del kernel de Linux. Este evento dejó a la comunidad sin una herramienta viable
para gestionar el desarrollo del proyecto.

Ante la retirada de la licencia gratuita por parte de BitMover y la presión
interna, Torvalds —a pesar de no querer realmente hacerlo— tomó la decisión de
crear su propia herramienta. En apenas dos semanas, desarrolló y lanzó Git, al
que describió con humor como 
\begin{quote}
\textbf{``el estúpido rastreador de contenidos"}
\end{quote}
\begin{quote}
\textbf{(“the stupid content tracker”)}
\end{quote}

Destacando su enfoque pragmático, minimalista y
descentralizado, en contraposición a las soluciones complejas y cerradas que
había conocido hasta entonces.

Fue en este contexto que Linus Torvalds diseñó y creó Git, un sistema de control
de versiones distribuido, rápido, escalable y completamente libre. Git no solo
solucionaba los problemas que habían afectado al desarrollo del kernel durante
años, sino que también establecía un nuevo estándar técnico y filosófico para la
colaboración en línea. Su arquitectura descentralizada, su manejo eficiente de
ramas y fusiones, y su integración fluida con el trabajo colaborativo hicieron
que Git se convirtiera rápidamente en el pilar de la infraestructura de
desarrollo moderno.


\subsection{La adopción masiva de Git}

Git terminó siendo tan increíblemente bueno en comparación con otros sistemas de
control de versiones que acabó comiéndose todo el mercado y teniendo una entidad
propia y muy reconocida. Antes de su aparición, no existía un flujo de trabajo
estandarizado en el mundo del software libre: cada proyecto y cada equipo
utilizaba sus propias herramientas, convenciones y procesos, lo que generaba una
auténtica `cacofonía' organizativa. En este contexto, Git no solo aportó una
solución técnica robusta, rápida y distribuida, sino que también instauró una
forma más estructurada y eficiente de colaborar en grandes comunidades,
transformando radicalmente la manera en que se desarrollaba software en todo el
mundo.

Git terminaría generando su mayor impacto a escala global cuando en 2008 se
funda la plataforma GitHub. Su impacto fue profundo y multifacético, entre sus
principales aportes se encuentran:

\begin{itemize}[label=$\bullet$, itemsep=0.5em]
    \item \textbf{Estandarización del flujo de trabajo:} GitHub consolidó la
    colaboración en una única plataforma durante el auge de la \hyperlink{web2}{Web 2.0}. Antes de
    su existencia, los proyectos usaban listas de correo y archivos comprimidos
    (\hyperlink{tarballs}{tarballs}) para compartir código, con procesos de colaboración poco
    uniformes. GitHub integró herramientas clave como gestión de issues,
    perfiles de contributores y pull requests, facilitando un flujo de trabajo
    limpio, transparente y colaborativo, accesible mediante una interfaz
    intuitiva.

    \item \textbf{Facilitación de la colaboración distribuida:} Al basarse en
    Git —el sistema de control de versiones distribuido creado por Linus
    Torvalds— GitHub permitió una colaboración verdaderamente descentralizada.
    Su plataforma hizo que Git fuese accesible incluso para desarrolladores sin
    experiencia previa, convirtiéndose rápidamente en el estándar de facto para
    el trabajo colaborativo a gran escala.

    \item \textbf{Visibilidad y descubrimiento de proyectos:} GitHub centralizó
    el ecosistema del código abierto, permitiendo a los usuarios descubrir
    nuevos proyectos y evaluar la reputación de sus contribuidores. Funciones
    como las estrellas ofrecieron indicadores simples de popularidad y calidad,
    impulsando la visibilidad y el crecimiento de iniciativas prometedoras.

    \item \textbf{Promoción de licencias permisivas:} GitHub también popularizó
    el uso de licencias abiertas como MIT y Apache, promoviendo una mayor
    reutilización del código. Esta tendencia contrastó con la filosofía más
    estricta del copyleft, ampliando el alcance del software libre y facilitando
    su adopción en contextos empresariales.

    \item \textbf{Cultura y comunidad del desarrollador:} Más allá de lo
    técnico, GitHub consolidó una nueva cultura alrededor del desarrollo de
    software: una comunidad global donde publicar código se volvió sinónimo de
    identidad profesional. Se transformó en la vitrina principal del trabajo de
    los desarrolladores, promoviendo una colaboración abierta y constante.
\end{itemize}

En conjunto, GitHub no solo modernizó la gestión de proyectos y la colaboración
técnica, sino que también redefinió la cultura del desarrollo de software en la
era del open source.

\subsection{La filosofía del código abierto y su difusión} 

Linus Torvalds ha sido una figura clave en la promoción del software de código
abierto, especialmente a través del desarrollo del kernel de Linux. Su decisión
de licenciar Linux bajo la GPL fue crucial, ya que permitió el acceso,
modificación y redistribución del código por parte de cualquier desarrollador.
Esta apertura fomentó la colaboración masiva y garantizó que las mejoras al
sistema también fueran compartidas libremente, desafiando así los modelos de
desarrollo propietarios y promoviendo una cultura basada en la transparencia y
la cooperación.

Linux demostró que el modelo de código abierto no solo era viable, sino también
competitivo frente al software propietario. Su adopción masiva impulsó nuevas
oportunidades comerciales, como servicios de soporte y mantenimiento, y atrajo
inversiones de grandes empresas como IBM. La adaptabilidad de Linux, impulsada
por su comunidad global, lo convirtió en una alternativa sólida a sistemas
cerrados. Linus Torvalds, por su parte, se mantuvo fiel a los principios del
software libre, rechazando ofertas muy lucrativas y priorizando siempre la utilidad
colectiva de su creación por encima del beneficio económico personal. 
El legado de Torvalds radica no solo en la creación de Linux y Git, sino también en la
demostración del poder de la transparencia y la colaboración en el desarrollo de
software. Su defensa del código abierto sentó un precedente y transformó la
industria del software, inspirando innumerables proyectos y fomentando una
cultura de compartición y mejora continua. 

\newpage
% Sección 7
\section{Conclusiones}
La trayectoria y las contribuciones de Linus Torvalds han dejado una marca
indeleble en la historia de la computación. Como creador del núcleo Linux y del
sistema de control de versiones distribuido Git, Torvalds no solo desarrolló
herramientas tecnológicas fundamentales, sino que también catalizó una
transformación profunda en la manera en que el software se desarrolla, comparte
y perfecciona.
La filosofía de código abierto promovida por Torvalds ha tenido un impacto
económico significativo, permitiendo la reducción de costos, fomentando la
innovación en hardware y software, y creando nuevos modelos de negocio basados
en servicios y soporte. Asimismo, ha generado un impacto social profundo,
democratizando el acceso a la tecnología, fomentando la colaboración global y
empoderando a los usuarios como co-creadores de su entorno digital.
Podemos decir sin ninguna duda que su trabajo ha sentado un precedente
fundamental que continúa moldeando el futuro de la tecnología, impulsando la
innovación abierta y la construcción colectiva del conocimiento digital.


\newpage

\begin{thebibliography}{9}

\bibitem{mochi_aleman_movimiento_2015}
Mochi Alemán, Prudencio Óscar, \textit{El movimiento del software libre},
Revista Mexicana de Ciencias Políticas y Sociales, vol. 45, no. 185, 2015.

\bibitem{stallman_free_2002}
Stallman, Richard y Stallman, Richard M., \textit{Free software, free society:
selected essays}, 1st ed., Free Software Foundation, Boston, Mass, 2002. ISBN:
978-1-882114-98-6.

\bibitem{campbell-kelly_computer_1996}
Campbell-Kelly, Martin y Aspray, William, \textit{Computer: a history of the
information machine}, BasicBooks, New York, 1996. ISBN: 978-0-465-02989-1.

\bibitem{torvalds_just_2002}
Torvalds, Linus y Diamond, David, \textit{Just for fun: the story of an
accidental revolutionary}, 1. HarperBusiness paperback ed., Harper, New York,
2002. ISBN: 978-0-06-662073-2.

\end{thebibliography}
\newpage

\subsection*{Definiciones de algunos términos} 

\begin{itemize} \item \hypertarget{amd}{\textbf{AMD:}} Fabricante de
procesadores y tecnologías relacionadas, competidor directo de Intel.
\label{amd}

\item \hypertarget{apple}{\textbf{Apple:}} Empresa conocida por computadoras
Macintosh y dispositivos como el iPhone. En los 90, usó variantes de UNIX en sus
sistemas y luego desarrolló macOS, basado en BSD.  \label{apple}

\item \hypertarget{assembler}{\textbf{Assembler:}} Lenguaje de bajo nivel para
programación directa del hardware. Usado en sistemas antiguos como ITS.
\label{assembler}

\item \hypertarget{asus}{\textbf{Asus:}} Fabricante taiwanés de hardware y
electrónica, incluyendo laptops y placas base.  \label{asus}

\item \hypertarget{att}{\textbf{AT\&T:}} Empresa de telecomunicaciones
estadounidense, pionera en computación y redes. Desarrolló UNIX en Bell Labs.
\label{att}

\item \hypertarget{aws}{\textbf{AWS (Amazon Web Services):}} Plataforma de
servicios en la nube utilizada globalmente para alojar y escalar aplicaciones.
\label{aws}

\item \hypertarget{basic}{\textbf{BASIC:}} Lenguaje de programación sencillo,
popular en microcomputadoras. Usado en entornos educativos.  \label{basic}

\item \hypertarget{belllabs}{\textbf{Bell Labs:}} Centro de investigación de
AT\&T donde se desarrolló UNIX. Cuna de innovaciones en computación y
telecomunicaciones.  \label{belllabs}

\item \hypertarget{bitkeeper}{\textbf{BitKeeper:}} Herramienta de control de
versiones propietaria usada temporalmente para Linux; su licencia motivó la
creación de Git.  \label{bitkeeper}

\item \hypertarget{bostos}{\textbf{BOS/TOS:}} Sistemas operativos tempranos para
mainframes, usados en entornos corporativos antes de la era de las PCs.
\label{bostos}

\item \hypertarget{bsd}{\textbf{BSD (Berkeley Software Distribution):}} Variante
de UNIX desarrollada en UC Berkeley. Base para sistemas como macOS.  \label{bsd}

\item \hypertarget{canonical}{\textbf{Canonical:}} Empresa que comercializa
Ubuntu, una de las distribuciones más populares de Linux.  \label{canonical}

\item \hypertarget{cobolfortran}{\textbf{COBOL/FORTRAN:}} Lenguajes de
programación para negocios (COBOL) y ciencia (FORTRAN). Dominantes en
mainframes.  \label{cobolfortran}

\item \hypertarget{concurrent versions system }{\textbf{Concurrent Versions
System (CVS):}} Sistema de control de versiones usado antes de Git y Subversion.
\label{concurrent versions system }

\item \hypertarget{corel}{\textbf{Corel:}} Empresa de software conocida por
CorelDRAW y otros productos de diseño gráfico.  \label{corel}

\item \hypertarget{copyleft}{\textbf{Copyleft:}} Filosofía legal que asegura que
las obras derivadas mantengan libertades de uso.  \label{copyleft}

\item \hypertarget{cpm}{\textbf{CP/M (Control Program for Microcomputers):}}
Sistema operativo de 8 bits, precursor de MS-DOS.  \label{cpm}

\item \hypertarget{debia}{\textbf{Debian:}} Distribución de Linux creada en 1993
por Ian Murdock. Destacada por su enfoque comunitario.  \label{debian}

\item \hypertarget{dell}{\textbf{Dell:}} Fabricante estadounidense de hardware,
especialmente computadoras personales y servidores.  \label{dell}

\item \hypertarget{digitalresearch}{\textbf{Digital Research:}} Empresa fundada
por Gary Kildall, creadora de CP/M.  \label{digitalresearch}

\item \hypertarget{dmca}{\textbf{DMCA (Digital Millennium Copyright Act):}} Ley
estadounidense que regula derechos de autor en el ámbito digital.  \label{dmca}

\item \hypertarget{fosdem}{\textbf{FOSDEM (Free and Open Source Developers'
European Meeting):}} Conferencia europea para desarrolladores de software libre.
\label{fosdem}

\item \hypertarget{ftp}{\textbf{FTP (File Transfer Protocol):}} Protocolo para
transferencia de archivos entre cliente y servidor.  \label{ftp}

\item \hypertarget{gcc}{\textbf{GCC (GNU Compiler Collection):}} Compiladores
libres para múltiples lenguajes como C y C++.  \label{gcc}

\item \hypertarget{github}{\textbf{GitHub:}} Plataforma basada en Git para
alojar y colaborar en proyectos de software.  \label{github}

\item \hypertarget{gnu}{\textbf{GNU (GNU's Not UNIX):}} Proyecto para crear un
sistema operativo libre compatible con UNIX.  \label{gnu}

\item \hypertarget{gpl}{\textbf{GPL (Licencia Pública General):}} Licencia
copyleft que garantiza libertad de uso, modificación y distribución.
\label{gpl}

\item \hypertarget{ibm}{\textbf{IBM (International Business Machines):}} Pionera
en tecnología. Desarrolló mainframes como el OS/360.  \label{ibm}

\item \hypertarget{intel}{\textbf{Intel:}} Fabricante líder de procesadores para
computadoras personales.  \label{intel}

\item \hypertarget{iot}{\textbf{IoT (Internet of Things):}} Conjunto de
dispositivos interconectados que recopilan e intercambian datos.  \label{iot}

\item \hypertarget{its}{\textbf{ITS (Incompatible Timesharing System):}} Sistema
operativo colaborativo del MIT.  \label{its}

\item \hypertarget{kernel}{\textbf{Kernel:}} Núcleo de un sistema operativo que
gestiona el hardware. Linux es un kernel.  \label{kernel}

\item \hypertarget{lenovo}{\textbf{Lenovo:}} Empresa china que fabrica hardware,
incluyendo laptops y servidores.  \label{lenovo}

\item \hypertarget{licencias mit/apache}{\textbf{Licencias MIT/Apache:}}
Licencias de código abierto permisivas. Requieren pocas condiciones para
redistribución.  \label{mitapache}

\item \hypertarget{mainframe}{\textbf{Mainframe:}} Computadora central de alto
rendimiento, utilizada en entornos empresariales.  \label{mainframe}

\item \hypertarget{meritocracia}{\textbf{Meritocracia Técnica:}} Modelo donde
las decisiones se basan en el mérito técnico de las contribuciones.
\label{meritocracia}

\item \hypertarget{minix}{\textbf{MINIX:}} Sistema operativo educativo en el que
Linus Torvalds se inspiró para desarrollar Linux.  \label{minix}

\item \hypertarget{microsoft}{\textbf{Microsoft:}} Empresa creadora de Windows y
MS-DOS. Representa el modelo de software propietario.  \label{microsoft}

\item \hypertarget{msdos}{\textbf{MS-DOS (Microsoft Disk Operating System):}}
Sistema operativo dominante en los 80, precursor de Windows.  \label{msdos}

\item \hypertarget{netscape}{\textbf{Netscape:}} Navegador popular en los 90. Su
liberación como código abierto dio origen a Mozilla Firefox.  \label{netscape}

\item \hypertarget{openstreetmap}{\textbf{OpenStreetMap:}} Proyecto colaborativo
para crear mapas libres con ayuda de la comunidad.  \label{openstreetmap}

\item \hypertarget{oracle}{\textbf{Oracle:}} Empresa de software conocida por
sus bases de datos relacionales y productos empresariales.  \label{oracle}

\item \hypertarget{os360}{\textbf{OS/360:}} Sistema operativo de IBM para
mainframes, influyente en el desarrollo del software moderno.  \label{os360}

\item \hypertarget{pascal}{\textbf{Pascal:}} Lenguaje estructurado usado
principalmente en educación y desarrollo inicial.  \label{pascal}

\item \hypertarget{perl}{\textbf{Perl:}} Lenguaje de programación flexible,
popular en administración de sistemas y desarrollo web.  \label{perl}

\item \hypertarget{php}{\textbf{PHP:}} Lenguaje de programación ampliamente
usado para el desarrollo web del lado del servidor.  \label{php}

\item \hypertarget{python}{\textbf{Python:}} Lenguaje de programación de alto
nivel, versátil y legible, popular en ciencia de datos, web y automatización.
\label{python}

\item \hypertarget{raspberrypi}{\textbf{Raspberry Pi:}} Computadora de bajo
costo usada en educación, domótica y proyectos de hardware libre.
\label{raspberrypi}

\item \hypertarget{redhat}{\textbf{Red Hat:}} Empresa que ofrece soporte
empresarial para Linux a través de Red Hat Enterprise Linux.  \label{redhat}

\item \hypertarget{richardstallman}{\textbf{Richard Stallman:}} Fundador del
proyecto GNU y de la Free Software Foundation.  \label{richardstallman}

\item \hypertarget{slackware}{\textbf{Slackware:}} Una de las distribuciones de
Linux más antiguas, conocida por su simplicidad y estabilidad.
\label{slackware}

\item \hypertarget{sun}{\textbf{Sun Microsystems:}} Conocida por desarrollar
Java y el sistema operativo Solaris.  \label{sun}

\item \hypertarget{suse}{\textbf{SUSE:}} Distribución de Linux alemana enfocada
en soluciones empresariales.  \label{suse}

\item \hypertarget{svn}{\textbf{SVN (Subversion):}} Sistema de control de
versiones sucesor de CVS, usado antes de Git.  \label{svn}

\item \hypertarget{tarballs}{\textbf{Tarballs:}} Archivos comprimidos usados
para distribuir código fuente de software en sistemas Unix/Linux.
\label{tarballs}

\item \hypertarget{valve}{\textbf{Valve:}} Empresa de videojuegos conocida por
Steam, impulsó el uso de Linux en el gaming.  \label{valve}

\item \hypertarget{web2}{\textbf{Web 2.0:}} Segunda generación de la web
centrada en la interacción del usuario y contenido generado por comunidades.
\label{web2} \end{itemize}


\end{document}

