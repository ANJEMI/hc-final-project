\documentclass[a4paper,12pt]{article}
\usepackage[utf8]{inputenc}
\usepackage[spanish]{babel}
\usepackage{amsmath}
\usepackage{graphicx}
\usepackage{tocbibind}
\usepackage{hyperref}

% \title{Linus Torvalds y el desarrollo de Linux}
% \author{José Antonio Concepción Alvarez\\  José Miguel Zayas Pérez \\ Grupo: 411 y 412}
% \date{\today}

\begin{document}

% Portada
% \maketitle

\begin{titlepage}
    \centering
    
    % Incluir la imagen
    \includegraphics[width=1.0\textwidth]{images/front.png} % Cambia la ruta a la ubicación de tu imagen
    
    \vspace{1cm}
    \newpage
    
    % Título
    {\Huge \textbf{Linus Torvalds y el desarrollo de Linux}}\\
    \vspace{0.5cm}
    
    % Autores
    {\Large José Antonio Concepción Alvarez \\  José Miguel Zayas Pérez \\ Grupo: 411 y 412}
    
    % \vfill
    
    % Fecha
    % {\large \today}
   \thispagestyle{empty} 
\end{titlepage}

\begin{centering}
    \vspace{2cm}
    \textbf{Resumen}\\
    Este ensayo explora la vida y las contribuciones de Linus Torvalds, el
    creador del núcleo Linux y Git, y su impacto en la historia de la
    computación. Se presenta un análisis de su relevancia en el desarrollo del
    software libre y de código abierto, destacando cómo su trabajo ha
    transformado el desarrollo colaborativo en la industria del software. A
    través de un examen de los antecedentes históricos, el nacimiento y
    evolución de Linux, y su influencia en la infraestructura de internet y
    dispositivos móviles, se argumenta que la influencia de Torvalds va más allá
    de la creación de software, abarcando un legado de colaboración y
    transparencia en el desarrollo tecnológico. Finalmente, se reflexiona sobre
    su impacto en el futuro de la tecnología y la filosofía del código abierto.
\end{centering}
\newpage

% Índice
\tableofcontents
\newpage

% Sección 1
\section{Introducción} - Presentación de Linus Torvalds y su relevancia en la
historia de la computación.\\ - Descripción general de sus contribuciones más
significativas: el núcleo Linux y Git.\\ - Planteamiento de la tesis: la
influencia de Torvalds va más allá de la creación de software, abarcando la
transformación del desarrollo colaborativo y el impulso del código abierto.\\ -
Breve esquema de los puntos que se abordarán en el trabajo.

% Sección 2
\section{Antecedentes} - Contexto histórico de la computación antes de la
aparición de Linux.\\ - El surgimiento del movimiento del software libre y de
código abierto.\\ - La trayectoria temprana de Linus Torvalds y su motivación
para desarrollar Linux.

% Sección 3
\section{El Nacimiento y Evolución de Linux} - El desarrollo inicial de Linux
como un proyecto personal.\\ - La transición de Linux a un proyecto colaborativo
global.\\ - El impacto de la Licencia Pública General de GNU (GPL) en la
expansión de Linux.\\ - La evolución de Linux y su adopción en servidores,
dispositivos móviles y sistemas embebidos.

% Sección 4
\section{El Impacto de Linux en la Industria de la Computación} - La influencia
de Linux en el desarrollo de servidores y la infraestructura de internet.\\ - El
papel de Linux en la revolución de los dispositivos móviles a través de
Android.\\ - La adopción de Linux en supercomputadoras y sistemas de alto
rendimiento.\\ - El impacto económico y social de Linux.

% Sección 5
\section{Linus Torvalds y el Desarrollo Colaborativo} - El modelo de desarrollo
de Linux y su influencia en la colaboración en línea.\\ - El papel de Torvalds
como líder y coordinador de la comunidad de Linux.\\ - El impacto de Git en la
gestión de versiones y la colaboración en proyectos de software.

% Sección 6
\section{La Filosofía del Código Abierto y su Difusión} - La defensa de Torvalds
del código abierto y su impacto en la industria del software.\\ - La influencia
de Linux en la adopción del código abierto en empresas y organizaciones.\\ - El
legado de Torvalds en la promoción de la transparencia y la colaboración en el
desarrollo de software.

% Sección 7
\section{Conclusiones} - Resumen de las contribuciones clave de Linus
Torvalds.\\ - Reafirmación de la tesis sobre su influencia en la computación.\\
- Reflexión sobre el legado de Torvalds y su impacto en el futuro de la
tecnología.

Bibliografía
\newpage
\begin{thebibliography}{9}
    \bibitem{referencia1}
    Autor, Título del libro, Editorial, Año.

    \bibitem{referencia2}
    Autor, Título del artículo, Revista, Año.
\end{thebibliography}

\end{document}

